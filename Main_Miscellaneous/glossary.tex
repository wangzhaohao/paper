% 如果论文中使用了大量的物理量符号、标志、缩略词、专门计量单位、自定义名词和术语等,应将全文中常用的这些符号及意义列出。如果上述符号和缩略词使用数量不多,可以不设专门的主要符号表,但在论文中出现时须加以说明。
% 论文中主要符号应全部采用法定单位,特别要严格执行GB3100~3102—93有关“量和单位”的规定。单位名称的书写,可以采用国际通用符号,也可以用中文名称,但全文应统一,不得两种混用。
% 缩略词应列出中英文全称。

% 注意 专有名词默认会将 希腊字母排在拉丁字母前,此处需要手动指定 `sort={}` 确保排序正确

\newglossaryentry{灌水器流量偏差系数}{name={\ensuremath{C_v}},description={灌水器流量偏差系数},}
\newacronym[description=计算机科学]{cs}{CS}{Computer Science}
\newacronym[description=逻辑卷管理器]{lvm}{LVM}{Logical Volume Manager}
\newglossaryentry{Linux}{
name=Linux,
description={is a generic term referring to the family of Unix-like
computer operating systems that use the Linux kernel},
plural=Linuces,
}
\newglossaryentry{管道内径}{name={\ensuremath{D}},description={管道内径/mm},sort={D}}
\newglossaryentry{灌水器流道当量直径}{name={\ensuremath{D_e}},description={灌水器流道当量直径/mm},sort={D_e}}
\newglossaryentry{管长}{name={\ensuremath{l}},description={管长/m},sort={l}}
\newglossaryentry{迷宫流道单元个数}{name={\ensuremath{n}},description={迷宫流道单元个数/个},sort={n}}
\newglossaryentry{灌水器流量}{name={\ensuremath{q}},description={灌水器流量/L·h-1},sort={q}}
\newglossaryentry{灌水器额定流量}{name={\ensuremath{q_n}},description={灌水器额定流量/L·h-1},sort={q_n}}
\newglossaryentry{雷诺数}{name={\ensuremath{Re}},description={雷诺数},sort={Re}}
\newglossaryentry{灌水器流量标准偏差}{name={\ensuremath{S_q}},description={灌水器流量标准偏差},sort={s_q}}
\newglossaryentry{流体的运动粘性系数}{name={\ensuremath{v}},description={流体的运动粘性系数},sort={v}}
\newglossaryentry{流态指数}{name={\ensuremath{X}},description={流态指数},sort={X}}
