% !TeX root = ../main.tex
\begin{chineseabstract}
\footnotetext{*本研究得到某某基金(编号:)的资助}
% \astfootnote{本研究得到某某基金(编号:)的资助}

论文摘要由摘要正文、关键词、论文类型、资助申明等部分组成。

博士学位论文摘要正文为1000字(word)左右,硕士学位论文摘要正文为600字(word)左右。
内容一般包括:从事这项研究工作的目的和意义;完成的工作(作者独立进行的研究工作及相应结果的概括性叙述);获得的主要结论(这是摘要的中心内容)。博士学位论文摘要应突出论文的创新点,硕士学位论文摘要应突出论文的新见解。

摘要中一般不用图、表、化学结构式、非公知公用的符号和术语。

如果论文的主体工作得到了有关基金资助,应在摘要第一页的页脚处标注:本研究得到某某基金(编号:)资助。

……

\chineseKeywords{XXX;XXX;XXX;XXX;XXX}

关键词由3~5个组成。关键词应从《汉语主题词表》中摘选,当《汉语主题词表》的词不足以反映主题时,可由申请人设计关键词,但须加注。每一关键词之间用分号分开,最后一个关键词后不打标点符号。由申请人设计的关键词,须在该关键词的右上角标注*,并在该页的页脚处注明“*表示非汉语主题词”。\newline

\chineseType{XXXX}

论文类型包括:a.理论研究(Theoretical Research);b.应用基础(Application Fundamentals);c.应用研究(Application Research);d.研究报告(Research Report);e.设计报告(Design Report);f.案例分析(Case Study);g.调研报告(Investigation Report);h.产品研发(Product Development);i.工程设计(Engineering Design);j.工程/项目管理(Engineering/Project Management);k.其它(Others)。

\end{chineseabstract}

